
\section*{Binary Search}

\begin{lstlisting}[language=C++]
#include <bits/stdc++.h>
using namespace std;

int binarySearch(vector<int>& arr, int low, int high, int x) {
	while (low <= high) {
		int mid = low + (high - low) / 2;
		
		// Verifica si x esta presente en mid
		if (arr[mid] == x)
		return mid;
		
		// Si x es mayor, ignorar la mitad izquierda
		if (arr[mid] < x)
		low = mid + 1;
		// Si x es menor, ignorar la mitad derecha
		else
		high = mid - 1;
	}
	
	// Si llegamos aqui, el elemento no estaba presente
	return -1;
}

int main() {
	vector<int> arr = {2, 3, 4, 10, 40};
	int x = 10;
	int result = binarySearch(arr, 0, arr.size() - 1, x);
	
	// Mostrar el resultado
	if (result == -1) {
		cout << "El elemento no esta presente en el arreglo." << endl;
	} else {
		cout << "El elemento esta presente en el indice " << result << "." << endl; // 0 indezado
	}
	return 0;
}
\end{lstlisting}
