
'\section*{Multiset}

\begin{lstlisting}[language=C++]
	#include <bits/stdc++.h>
	using namespace std;
	
	int main() {
		multiset<int> ms; // []
		
		// Insertar elementos
		ms.insert(10); // [10]
		ms.insert(20); // [10, 20]
		ms.insert(10); // [10, 10, 20]
		ms.insert(30); // [10, 10, 20, 30]
		ms.insert(20); // [10, 10, 20, 20, 30]
		ms.insert(40); // [10, 10, 20, 20, 30, 40]
		ms.insert(20); // [10, 10, 20, 20, 20, 30, 40]
		ms.insert(50); // [10, 10, 20, 20, 30, 40, 50]
		
		// Acceder e imprimir elementos
		for (const auto& elem : ms) {
			cout << elem << " ";
		}
		cout << "\n";
		
		// Obtener el primer elemento
		cout << *ms.begin() << "\n"; // 10
		
		// Obtener el ultimo elemento
		cout << *prev(ms.end()) << "\n"; // 50
		
		// Usar count para verificar la existencia y la cantidad de un elemento
		cout << ms.count(20) << "\n"; // 3
		cout << ms.count(60) << "\n"; // 0
		
		// Eliminar un elemento (elimina solo una instancia si hay duplicados)
		auto it = ms.find(20); 
		if(it != ms.end()){
			ms.erase(it); // [10, 10, 20, 20, 30, 40, 50]
		}
		
		// Eliminar todas las instancias de un elemento
		ms.erase(10); // [20, 20, 30, 40, 50]
		
		// Verificar existencia de un elemento despues de eliminar
		if (ms.find(10) == ms.end()) {
			cout << "Elemento no encontrado.\n"; 
		}
		
		cout << ms.size() << "\n"; // 4
		
		// Vaciar el multiconjunto
		ms.clear();
		
		cout << (ms.empty() ? "Si" : "No") << "\n"; // Si
		
		return 0;
	}
	
\end{lstlisting}
